\documentclass[../main.tex]{subfiles}

\begin{document}
	

\section{Microcontainer Update Manager Demo}

When building the example application for the native target, the firmware update
capability is removed. Instead two in-memory slots are created that can be
updated with new payloads. Both of these in-memory slots are hooked up to the
Femto-Container CoAP endpoints. These act as a demonstrator for the SUIT
capabilities together with Femto-Containers.

The steps described here show how to use SUIT manifests to deliver content
updates to a RIOT instance. The full workflow is described, including the setup
of simple infrastructure.

\subsection{Prerequisites}

Install python dependencies (only Python3.6 and later is supported):

\begin{lstlisting}
pip3 install --user cbor2 cryptography
\end{lstlisting}

Install aiocoap from the source

\begin{lstlisting}
pip3 install --user aiocoap[linkheader]>=0.4.1
\end{lstlisting}

See the aiocoap installation instructions
more details.  add \texttt{~/.local/bin} to PATH The aiocoap tools are
installed to \texttt{~/.local/bin}. Either add \texttt{export
PATH=\$PATH:~/.local/bin} to your \texttt{~/.profile} and re-login, or execute
that command *in every shell you use for this tutorial*.
	
\subsection{Key Management}

SUIT keys consist of a private and a public key file, stored in \texttt{\$(SUIT\_KEY\_DIR)}.
Similar to how ssh names its keyfiles, the public key filename equals the
private key file, but has an extra \texttt{.pub} appended.
\texttt{SUIT\_KEY\_DIR} defaults to the \texttt{keys/} folder at the top of a RIOT checkout.
If the chosen key doesn't exist, it will be generated automatically.
That step can be done manually using the \texttt{suit/genkey} target.

\subsection{Setting up networking}

To deliver the payload to the native instance, a network connection between a
coap server and the instance is required. First a bridge with two tap devices is created:

\begin{lstlisting}
sudo RIOT/dist/tools/tapsetup/tapsetup -c
\end{lstlisting}

This creates a bridge called `tapbr0` and a `tap0` and `tap1`. These last two
tap devices are used by native instances to inject and receive network packets
to and from.

On the bridge device `tapbr0` an routable IP address is added such as
\texttt{2001:db8::1/64}:

\begin{lstlisting}
sudo ip address add 2001:db8::1/64 dev tapbr0
\end{lstlisting}

\subsection{Starting the CoAP server}

As mentioned above, a CoAP server is required to allow the native instance to
retrieve the manifest and payload. The `aiocoap-fileserver` is used for this,
hosting files under the `coaproot` directory:

\begin{lstlisting}
aiocoap-fileserver coaproot
\end{lstlisting}

This should be left running in the background. A different directory can be used
if preferred.

\subsection{Building and starting the example}

Before the natice instance can be started, it must be compiled first.
Compilation can be started from the root of your RIOT directory with:

\begin{lstlisting}
make -C suit_femtocontainer
\end{lstlisting}

Then start the example with:

\begin{lstlisting}
make -C suit_femtocontainer term
\end{lstlisting}

This starts an instance of the \texttt{suit\_update} example as a process on your
computer. It can be stopped by pressing \texttt{ctrl+c} from within the application.

The instance must also be provided with a routable IP address in the same range
configured on the `tapbr0` interface on the host. In the RIOT shell, this can be
done with:

\begin{lstlisting}
> ifconfig 5 add 2001:db8::2/64
\end{lstlisting}

Where 5 is the interface number of the interface shown with the `ifconfig`
command.

\subsection{Generating the femto-container application and manifest}

To update the storage location we first need the Femto-Container payload
application:

\begin{lstlisting}
make -C suit_femtocontainer/bpf
cp suit_femtocontainer/bpf/temp_sens.bin
\end{lstlisting}

Make sure to store it in the directory selected for the CoAP file server.

Next, a manifest template is created. This manifest template is a JSON file that
acts as a template for the real SUIT manifest. Within RIOT, the script
\texttt{dist/tools/suit/gen\_manifest.py} is used.

\begin{lstlisting}
dist/tools/suit/gen_manifest.py --urlroot coap://[2001:db8::1]/
	--seqnr 1 -o suit.tmp coaproot/temp_sens.bin:0:ram:0
\end{lstlisting}

This generates a suit manifest template with the sequence number set to \texttt{1}, a
payload that should be stored at slot offset zero in slot \texttt{.ram.0}. The url for
the payload starts with \texttt{coap://[fe80::4049:bfff:fe60:db09]/}. Make sure to
match these with the locations and IP addresses used on your own device.

SUIT supports a check for a slot offset. Within RIOT this is normally used to
distinguish between the different firmware slots on a device. As this is not
used on a native instance, it is set to zero here. The location within a SUIT
manifest is an array of path components. Which character is used to separate
these path components is out of the scope of the SUIT manifest. The
\texttt{gen\_manifest.py} command uses colons (\texttt{:}) to separate these components.
Within the manifest this will show up as an array containing \texttt{[ "ram", "0" ]}.

The content of this template file should look like this:

\begin{lstlisting}
{
    "manifest-version": 1,
    "manifest-sequence-number": 1,
    "components": [
        {
            "install-id": [
                "ram",
                "0"
            ],
            "vendor-id": "547d0d746d3a5a9296624881afd9407b",
            "class-id": "bcc90984fe7d562bb4c9a24f26a3a9cd",
            "file": "coaproot/temp_sens.bin",
            "uri": "coap://[fe80::4049:bfff:fe60:db09]/temp_sens.bin",
            "bootable": false
        }
    ]
}
\end{lstlisting}

The manifest version indicates the SUIT manifest specification version numbers,
this will always be 1 for now. The sequence number is the monotonically
increasing anti-rollback counter.

Each component, or payload, also has a number of parameters. The install-id
indicates the unique path where this component must be installed.
The vendor and class ID are used in manifest conditionals to ensure that the
payload is valid for the device it is going to be installed in. It is generated
based on the UUID(v5) of `riot-os.org` and the board name (`native`).

The file and uri are used to generated the URL parameter and the digest in the
manifest. The bootable flag specifies if the manifest generator should instruct
the node to reboot after applying the update.

Generating the actual SUIT manifest from this is done with:

\begin{lstlisting}
dist/tools/suit/suit-manifest-generator/bin/suit-tool create -f suit
	-i suit.tmp -o coaproot/suit\_manifest
\end{lstlisting}

This generates the manifest in SUIT CBOR format. The content can be inspected by
using the `parse` subcommand:

\begin{lstlisting}
dist/tools/suit/suit-manifest-generator/bin/suit-tool parse
	-m coaproot/suit\_manifest
\end{lstlisting}

The manifest generated doesn't have an authentication wrapper, it is unsigned
and will not pass inspection on the device or RIOT instance. The manifest can be
signed with the `sign` subcommand together with the keys generated earlier.

\begin{lstlisting}
dist/tools/suit/suit-manifest-generator/bin/suit-tool sign
	-k keys/default.pem -m coaproot/suit\_manifest
	-o coaproot/suit\_manifest.signed
\end{lstlisting}

This generates an authentication to the manifest. This is visible when
inspecting with the `parse` subcommand. The URL to this signed manifest will be
submitted to the instance so it can retrieve it and in turn retrieve the
component payload specified by the manifest.

\subsection{Updating the storage location}

The update process is a two stage process where first the instance pulls in the
manifest via a supplied url. It will download the manifest and verify the
content. After the manifest is verified, it will proceed with executing the
command sequences in the manifest and download the payload when instructed to.

The URL for the manifest can be supplied to the instance via the command line.

\begin{lstlisting}
> suit coap://[2001:db8::1]/suit_manifest.signed
\end{lstlisting}

The payload is the full URL to the signed manifest. The native instance should
respond on this by downloading and executing the manifest. If all went well, the
output of the native instance should look something like this:

\begin{lstlisting}
suit coap://[2001:db8::1]/suit_manifest.signed
suit_coap: trigger received
suit_coap: downloading "coap://[2001:db8::1]/suit_manifest.signed"
suit_coap: got manifest with size 276
suit: verifying manifest signature
suit: validated manifest version
Retrieved sequence number: 0
Manifest seq_no: 1, highest available: 0
suit: validated sequence number
Formatted component name: .ram.0
validating vendor ID
Comparing 547d0d74-6d3a-5a92-9662-4881afd9407b
	to 547d0d74-6d3a-5a92-9662-4881afd9407b from manifest
validating vendor ID: OK
validating class id
Comparing bcc90984-fe7d-562b-b4c9-a24f26a3a9cd
	to bcc90984-fe7d-562b-b4c9-a24f26a3a9cd from manifest
validating class id: OK
SUIT policy check OK.
Formatted component name: .ram.0
Fetching firmware |█████████████████████████| 100%
Finalizing payload store
Verifying image digest
Starting digest verification against image
Install correct payload
Verifying image digest
Starting digest verification against image
Install correct payload
\end{lstlisting}

The storage location can now be inspected using the built-in command. If the
same payload as suggested above was used, it should look like this:

\begin{lstlisting}
> storage_content .ram.0 0 64
72425046000000000000...
\end{lstlisting}

The process can be done multiple times with both slot `.ram.0` and `.ram.1` and
different payloads. Keep in mind that the sequence number is a strict
monotonically number and must be increased after every update.

subsection{Femto-Containers}

The Femto-Containers executable is stored in slot \texttt{.ram.0} for simplicity in
this example. It can be triggered for execution by sending a POST request over
CoAP to the texttt{/bpf/exec/0} endpoint on the instance running.

\begin{lstlisting}
aiocoap-client -m POST coap://[2001:db8::2]/bpf/exec/0
\end{lstlisting}

\end{document}
